\documentclass[a4paper, 12pt]{article}
\usepackage[utf8]{inputenc}
\renewcommand\familydefault{\sfdefault}
\usepackage[T1]{fontenc}
\usepackage[francais]{babel}
\usepackage[left=2cm,top=2cm,right=2cm,bottom=2cm]{geometry}
\usepackage{graphicx}
\usepackage{minted}
\usemintedstyle{colorful}
\usepackage{float}
\floatplacement{figure}{H}
\usepackage{authblk}
\usepackage{enumitem}
\usepackage{hyperref}
\hypersetup{
    colorlinks,
    citecolor=black,
    filecolor=black,
    linkcolor=black,
    urlcolor=blue
}

\usepackage{caption}
\newenvironment{code}{\captionsetup{type=listing}}{}

\begin{document}

\title{Vagrant et Ansible}
\author{Raed Abdennadher, Jean-Etienne Delucinge, Steven Liatti, Gaëtan Ringot}
\affil{\small Cours de Virtualisation des SI - Prof. Massimo Brero}
\affil{\small Hepia ITI 3\up{ème} année}
\maketitle

\begin{figure}
	\begin{center}
		\includegraphics[width=0.6\textwidth]{intro.jpg}
	\end{center}
\end{figure}
\newpage

\tableofcontents
\renewcommand\listoflistingscaption{Table des scripts de code}
\listoflistings

\newpage


\section{Vagrant}
Nous (Raed et Steven) avons commencé à explorer \href{https://www.vagrantup.com/}{Vagrant} avec
\href{https://www.linux-kvm.org/page/Main_Page}{KVM} sous l'impulsion du groupe Cluster KVM
(Sylvain, Loïc, Nathanaël et Mayron). Ils aimeraient pouvoir déployer et configurer leurs VMs automatiquement.

\subsection{Qu'est-ce que Vagrant ?}
Vagrant est un outil de gestion du cycle de vie de machines virtuelles.
Il permet de créer et de lancer une ou plusieurs machines virtuelles pré-configurées
selon des critères définis à l'avance dans des fichiers de configuration (Vagrantfile).

\subsection{Recherches et tests avec Vagrant}
Le problème initial est que Vagrant peut déployer des machines virtuelles uniquement pour
VirtualBox, Hyper-V et VMware par défaut. Nous avons alors cherché un moyen de déployer
pour KVM. Un plugin/librairie existe, \href{https://github.com/vagrant-libvirt/vagrant-libvirt}{vagrant-libvirt}
(provider pour Vagrant). Nous avons commencé la lecture de cette doc. Nous avons utilisé
la version 1.9.1 de Vagrant, qui n'est pas la dernière en date et qui est compatible avec
libvirt. Nous avons réalisé toutes ces manipulations sur nos machines personnelles car
la plupart du temps les droits \mintinline{text}{root} étaient requis et qu'il n'y avait
pas forcément les bonnes versions des softs.
​Nous avons eu de la peine à utiliser Vagrant avec libvirt et KVM car Vagrant n'est pas
compatible par défaut avec KVM. Nous avons cherché plusieurs tutoriels et sommes tombés
sur cet unique script qui avait l'air prometteur. Avec ce script nous voulions
transformer notre image \mintinline{text}{client1} existante pour la donner au groupe
KVM (voir ligne 12 dans le code suivant, \mintinline{text}{centos-7-client1-disk1.vmdk}).
\begin{code}
    \inputminted[breaklines,breaksymbol=,linenos,frame=single,stepnumber=1,tabsize=2]{bash}{../vagrant/libvirt/script.sh}
    \caption{Script Vagrant et libvirt}
    \label{script_libvirt}
\end{code}
\bigbreak
​Ce script a fonctionné partiellement, car il ne prenait pas l'image qu'on lui a fournit,
mais une image par défaut. \textbf{C'est pour cela que nous avons abandonné l'idée d'utiliser
Vagrant avec KVM et libvirt}. Nous avons toutefois remarqué (\textbf{27.11.2017}) que la librairie
a été mise à jour ces derniers jours et, qu'apparemment, elle est compatible avec la dernière version
de Vagrant (2.0.1).

\subsection{Commandes et scripts pour CentOS 7}
Vagrant utilise des "box" comme images de base, qu'on peut lui fournir en local (typiquement des fichiers
iso ou .img) ou depuis \href{https://app.vagrantup.com/boxes/search}{Vagrant Cloud}. Dans notre cas, nous avons
utilisé l'image \mintinline{text}{centos/7}. Voici une marche à suivre pour utiliser Vagrant avec VirtualBox.
Tout d'abord, nous créons un dossier pour Vagrant :
\begin{code}
    \begin{minted}[breaklines,breaksymbol=,linenos,frame=single,stepnumber=1,tabsize=2]{bash}
$ mkdir vagrant_folder
$ cd vagrant_folder
$ vagrant init
    \end{minted}
    \caption{vagrant init}
    \label{vagrantinit}
\end{code}
\bigbreak
\mintinline{text}{vagrant init} crée un fichier de configuration nommé \mintinline{text}{Vagrantfile},
dont voici le contenu initial :
\begin{code}
    \begin{minted}[breaklines,breaksymbol=,linenos,frame=single,stepnumber=1,tabsize=2]{ruby}
Vagrant.configure("2") do |config|
  config.vm.box = "base"
end
    \end{minted}
    \caption{Vagrantfile}
    \label{Vagrantfile}
\end{code}
\bigbreak
Ensuite, nous exécutons la commande \mintinline{bash}{vagrant box add centos/7}, qui télécharge la box
en question et nous permet de choisir le provider avec lequel nous allons travailler. Une fois cette tâche
terminée (téléchargement), devons normalement obtenir le message suivant :
\begin{code}
    \begin{minted}[breaklines,breaksymbol=,linenos,frame=single,stepnumber=1,tabsize=2]{bash}
==> box: Successfully added box 'centos/7' (v1710.01) for 'VirtualBox'!
    \end{minted}
\end{code}
\bigbreak
Nous pouvons maintenant commencer à éditer notre fichier \mintinline{text}{Vagrantfile} comme ceci :
\begin{code}
    \begin{minted}[breaklines,breaksymbol=,linenos,frame=single,stepnumber=1,tabsize=2]{ruby}
Vagrant.configure("2") do |config|
  config.vm.box = "centos/7"
end
    \end{minted}
    \caption{Vagrantfile centos}
    \label{Vagrantfilecentos}
\end{code}
\bigbreak
Finalement, nous lançons la machine virtuelle avec la commande suivante : \mintinline{bash}{vagrant up}.
Nous constatons que par défaut le port pour SSH est le 2222, que l'IP est la 127.0.0.1 et que
\mintinline{text}{username = vagrant} et utilise une paire de clé pour l'authentification. Finalement nous
pouvons nous connecter à la machine virtuelle avec la commande suivante : \mintinline{bash}{vagrant ssh}


\subsection{Déploiement et configuration}
Nous avons continué nos expériences avec Vagrant avec VirtualBox comme provider. Nous avons commencé par
essayer avec une image CentOS, mais arrivés à l'étape de la configuration du serveur SSH, nous n'avons pas
réussi à nous connecter. D'après l'erreur survenue
(\textit{ssh\_exchange\_identification: read: connection reset by peer}) et
\href{https://unix.stackexchange.com/questions/151860/ssh-exchange-identification-read-connection-reset-by-peer/151866#151866}{nos recherches sur le web},
il semblerait que ce soit une incompatibilité entre le client SSH des machines hepia et du serveur SSH CentOS.
Nous avons finalement réussi à installer, configurer et lancer une machine virtuelle sous Debian (Jessie) et pouvoir
s'y connecter en SSH avec l'utilisateur \mintinline{text}{vagrant}. Voici les fichiers de configuration et les scripts.
\bigbreak
Pour lancer toute la séquence, il faut exécuter \mintinline{text}{vagrant.sh}. Nous allons maintenant décrire la séquence
complète et dans l'ordre d'exécution.
\begin{code}
    \begin{minted}[breaklines,breaksymbol=,linenos,frame=single,stepnumber=1,tabsize=2]{text}
.
|-- debian
|   |-- bootstrap.sh
|   |-- conf_files_vm
|   |   |-- apt.conf
|   |   |-- interfaces
|   |   |-- proxy
|   |   |-- pub_key_server_ansible
|   |   |-- resolv.conf
|   |   |-- sshd_config
|   |-- create.sh
|   |-- Vagrantfile
|-- vagrant.sh
    \end{minted}
    \caption{Arborescence des fichiers pour Vagrant}
    \label{Arborescence}
\end{code}
\bigbreak

​Ce fichier initial (\ref{vagrant.sh}) va copier le dossier \mintinline{text}{<template>} de l'OS voulu pour la VM dans le dossier
\mintinline{text}{<folder_name>}. En effet, si on désire créer plusieurs machines virtuelles, il faut créer un
dossier pour vagrant par machine virtuelle. Ce script va appeler le script suivant, \mintinline{text}{create.sh},
dans le dossier \mintinline{text}{<folder_name>}, en lui passant le nom de la VM, son IP, le nombre de CPU et la
quantité de RAM voulus. ​Finalement, une fois que la VM a été configurée et redémarrée, on lance le script Ansible
depuis la VM serveur Ansible (cf. doc Ansible). Cette dernière étape est nécessaire uniquement car nous n'avons pas
les droits pour installer Ansible sur la même machine physique de l'école où est installé Vagrant.
\begin{code}
    \inputminted[breaklines,breaksymbol=,linenos,frame=single,stepnumber=1,tabsize=2]{bash}{../vagrant/vagrant.sh}
    \caption{vagrant.sh}
    \label{vagrant.sh}
\end{code}
\bigbreak

Script de création (\ref{create.sh}) : on commence par modifier le fichier \mintinline{text}{interfaces} et \mintinline{text}{Vagrantfile}
avec les arguments fournis (IP, nom, CPU et RAM) avec l'utilitaire \mintinline{text}{sed} (lignes 21 à 24).
On continue par supprimer si besoin une éventuelle image ayant le même nom (ligne 26), puis on lance l'installation
(appel à \mintinline{text}{Vagrantfile}) (ligne 27) et une fois l'installation terminée, nous arrêtons la machine
(ligne 28). Les deux dernières lignes concernent VirtualBox, la 1ère permet à la machine virtuelle d'accéder au LAN
(mode bridge) et la dernière redémarre la VM.
\begin{code}
    \inputminted[breaklines,breaksymbol=,linenos,frame=single,stepnumber=1,tabsize=2]{bash}{../vagrant/debian/create.sh}
    \caption{create.sh}
    \label{create.sh}
\end{code}
\bigbreak

Ce fichier (\ref{Vagrantfiledebian}) est le fichier de configuration d'une VM par Vagrant. Sont définis le nom de l'image (téléchargée sur
\href{https://app.vagrantup.com/boxes/search}{Vagrant Cloud}) (ligne 2), le script shell qui sera exécuté à la fin
de l'installation (ligne 3), le provider et le nom de la VM (lignes 3 et 4) et la RAM et le CPU (lignes 6 et 7).
​À noter que tout le contenu du dossier où se trouve ce fichier \mintinline{text}{Vagrantfile} sera copié à la
racine de la VM, dans le dossier \mintinline{text}{/vagrant}.
\begin{code}
    \inputminted[breaklines,breaksymbol=,linenos,frame=single,stepnumber=1,tabsize=2]{ruby}{../vagrant/debian/Vagrantfile}
    \caption{Vagrantfile debian}
    \label{Vagrantfiledebian}
\end{code}
\bigbreak

Script (\ref{bootstrap.sh}) exécuté lorsque l'installation de la machine virtuelle est terminée, par elle-même.
Nous copions tous les fichiers de configuration pré-écrits et mettons à jour les applications
installées, ceci faisant également office de test de connectivité internet. La clé publique est
celle du serveur Ansible.
\begin{code}
    \inputminted[breaklines,breaksymbol=,linenos,frame=single,stepnumber=1,tabsize=2]{bash}{../vagrant/debian/bootstrap.sh}
    \caption{bootstrap.sh}
    \label{bootstrap.sh}
\end{code}
\bigbreak

\newpage
Fichier de configuration des interfaces réseau pour Debian.
\begin{code}
    \inputminted[breaklines,breaksymbol=,linenos,frame=single,stepnumber=1,tabsize=2]{text}{../vagrant/debian/conf_files_vm/interfaces}
    \caption{interfaces}
    \label{interfaces}
\end{code}
\bigbreak

Configuration du proxy (contrainte du réseau de l'hepia).
\begin{code}
    \inputminted[breaklines,breaksymbol=,linenos,frame=single,stepnumber=1,tabsize=2]{text}{../vagrant/debian/conf_files_vm/proxy}
    \caption{proxy}
    \label{proxy}
\end{code}
\bigbreak

Définition du proxy pour \mintinline{text}{apt} également.
\begin{code}
    \inputminted[breaklines,breaksymbol=,linenos,frame=single,stepnumber=1,tabsize=2]{text}{../vagrant/debian/conf_files_vm/apt.conf}
    \caption{apt.conf}
    \label{apt.conf}
\end{code}
\bigbreak

Fichier de configuration DNS (mauvaise adresse par défaut).
\begin{code}
    \inputminted[breaklines,breaksymbol=,linenos,frame=single,stepnumber=1,tabsize=2]{text}{../vagrant/debianresolvconf_files_vm/resolv.conf}
    \caption{resolv.conf}
    \label{resolv.conf}
\end{code}
\bigbreak

Configuration du serveur SSH sur Debian. Le seul changement qui a été fait par rapport à la version
par défaut inclue dans cette image Debian se trouve à la dernière ligne,
\mintinline{text}{PasswordAuthentication yes} au lieu de \mintinline{text}{PasswordAuthentication no}, pour autoriser
la connexion SSH avec mot de passe.
\begin{code}
    \inputminted[breaklines,breaksymbol=,linenos,frame=single,stepnumber=1,tabsize=2]{text}{../vagrant/debian/conf_files_vm/sshd_config}
    \caption{sshd\_config}
    \label{sshd_config}
\end{code}
\bigbreak

Une fois que toutes ces étapes sont terminées, nous pouvons nous connecter à cette machine
virtuelle (depuis le réseau hepia), par cette commande et avec mot de passe \textbf{vagrant} :
\mintinline{bash}{ssh vagrant@10.194.184.xxx}.





\section{Ansible}
\subsection{Introduction}
\subsubsection{Qu'est-ce que Ansible ?}
Ansible est un logiciel libre d'automatisation des applications et de l'infrastructure informatique.
Déploiement d'application, gestion de Configuration et livraison en continue.
\subsubsection{But final}
Nous (Gaetan et Jean-Etienne) avons commencé à explorer Ansible et ces possibilités pour pouvoir
déployer des installations, mises à jour, ... automatiquement dans un réseau pour pouvoir l'intégrer
avec Vagrant (Read et Steven). Au final nous devons déployer une machine virtuelle avec un service et
des paramètres donner.

\subsubsection{Version utilisée (Ansible et OS)}
Pour notre projet nous avons installés un CentOS 7 avec une configuration minimale, dans laquelle nous avons par la suite ajouté :
\begin{itemize}
    \item EPEL release
    \item Ansible (version 2.4.1.0)
    \item Python (version 2.7.5)
    \item Nano
\end{itemize}
\textbf{Attention}, la mise à jour d'Ansible peut fortement créer des soucis de compatibilités.

\subsubsection{Format à respecter impérativement en cas de création d'un fichier YAML (.yml)}
\begin{itemize}
    \item Chaque fichier doit commencer par "\textbf{---}" (3 tirets normaux) hors rôle
    \item L'indentation est très importante, utiliser uniquement des \textbf{ESPACES}
    \item Suivant l'OS, il est important de les différencier, car les commandes ne sont pas multiplateformes et entraine la non-fonctionnalité du script ansible
\end{itemize}

\subsection{Accès SSH sur les machines}
Si des problèmes de connexion en ssh arrivent , utiliser les commandes suivantes :
\begin{code}
    \begin{minted}[breaklines,breaksymbol=,linenos,frame=single,stepnumber=1,tabsize=2]{bash}
$ ssh-add
$ ssh-keygen -t rsa -C "ansible@ip_ansible"
$ ssh-copy-id user@ipduclient
    \end{minted}
    \caption{SSH Ansible}
    \label{ssh_ansible}
\end{code}
\bigbreak
Permettent de régler les problèmes en créant puis copiant des clés SSH afin de par la suite, ne pas avoir de problèmes d'identification.
\textbf{À noter} : la génération de clés n'est pas nécessaires si déjà effectué une fois.

\subsection{Principe des groupes d'utilisateurs/machines}
Il est à noter que le fichier se situant dans le dossier \mintinline{text}{/etc/ansible/hosts} se trouvent les
différents groupes de machines et utilisateurs pour ansible.
Un utilisateur PEUT être dans PLUSIEURS groupes différents. Lors d'une commande ansible, il est possible de faire
appel à un certain groupe, en rajoutant celui-ci à la fin de la commande ansible.
Exemple avec le groupe \textbf{Users} : \mintinline{text}{ansible -m ping Users}
\bigbreak
Afin de rajouter un groupe, il faut reprendre la syntaxe suivante :
\begin{code}
    \begin{minted}[breaklines,breaksymbol=,linenos,frame=single,stepnumber=1,tabsize=2]{text}
[nom_groupe]
nom_host1 ansible_ssh_host= addresse ip 1
nom_host2 ansible_ssh_host= addresse ip 2
nom_host3 ansible_ssh_host= addresse ip 3
    \end{minted}
    \caption{Utilisateurs Ansible}
    \label{Utilisateurs_ansible}
\end{code}
\bigbreak
Mais attention, il faut penser à créer le fichier \mintinline{text}{/etc/ansible/group_vars/nom_groupe}.
Il contient toutes les informations de configuration de connexion, comme ansible\_ssh\_user, qui n'est autre
que le nom d'utilisateur utilisé pour se connecter au groupe en question.



\subsection{Utilisation de playbooks}
Pour lancer un playbook dans Ansible il faut exécuter la commande suivante :
\\
\mintinline{text}{ansible-playbook chemin_du_fichier/Nom_du_fichier -e "nom_variable=valeur_var"}.
\bigbreak
Valeurs importantes à mettre dans le fichier en question :
\begin{itemize}
    \item hosts : Qui indique  le groupe ou la machine qui exécutera nos futures tâches/services
    \item remote\_user : nom d'utilisateur utilisé pour exécuter les tâches en question.
    \item tasks : Indiquant le début de la liste des tâches qui vont être exécutées par la suite.
\end{itemize}
\textbf{Attention !} : hosts a priorité sur remote\_user !
\bigbreak
Exemple simple avec un ping :
\begin{code}
    \begin{minted}[breaklines,breaksymbol=,linenos,frame=single,stepnumber=1,tabsize=2]{yaml}
---
- hosts: Users
  remote_user: root

  tasks:
    - ping:
    \end{minted}
    \caption{Ping}
    \label{Ping}
\end{code}
\bigbreak

Dans le cadre d'un service, celui-ci doit être appelé grâce à la façon suivante :
\begin{code}
    \begin{minted}[breaklines,breaksymbol=,linenos,frame=single,stepnumber=1,tabsize=2]{yaml}
- service:
     name: Nom_du_service_à_exécuter
     state: Etat_futur_du_service (started, stopped, restarted, reloaded)
    \end{minted}
    \caption{Service}
    \label{Service}
\end{code}
\bigbreak
\bigbreak
Il est possible de récupérer des variables mises avec l'option -e (voir plus haut) pour ce faire le formatage du fichier se fait de la façon suivante :
\begin{itemize}
    \item Nom\_variable : "Nom\_variable" = peut récupérer une liste/tableau
    \item Nom\_variable : "Nom\_variable" = peut récupérer une string
\end{itemize}
\begin{code}
    \begin{minted}[breaklines,breaksymbol=,linenos,frame=single,stepnumber=1,tabsize=2]{yaml}
---
  vars:
      Nom_variable: "{{ Nom_variable }}"

  tasks:
- include: tasks1.yml
when: 1

- include: tasks2.yml
when: 2
    \end{minted}
    \caption{Variables}
    \label{Variables}
\end{code}
\bigbreak

\subsubsection{Condition When dans ansible}
Avec une condition when on peut faire un "if" avec une condition, ce qui permet d'appelle des roles/playbook ,
de faire des tâches et etc selon une condition.

\subsubsection{Pourquoi on l'utilise ?}
Nous avons utilisé la condition "ansible\_os\_family" (intégrer dans les services d'ansible) pour différencier par
rapport à la famille de l'OS pour utiliser les bonnes commandes liées à celui-ci. Nous avons aussi utilisé pour
notre variable "service" qui permet de savoir quel service installer (Web,dhcp,..).

\subsubsection{Rôles (extrait de buzut.fr)}
Les rôles représentent une manière d'abstraire les directives includes. Grâce aux rôles, il n'est plus utile de
préciser les divers includes dans le playbook, ni les paths des fichiers de variables etc.​	Le playbook n'a qu'à
lister les différents rôles à appliquer. En outre, depuis les tasks du rôle, l'ensemble des chemins sont relatifs.
Inutile donc de préciser le nom du fichier suffit, Ansible s'occupe du reste.

\subsubsection{Pourquoi un/des rôle(s)}
Car cela nous permet de généraliser une installation par service (web,dhcp,..) plus facilement et réutilisable
par d'autres personnes !

\subsubsection{Génération du rôle}
Pour générer un rôle , il faut utiliser "ansible-galaxy + nom + init". Il va générer les dossiers/fichiers de
base pour notre rôle. Notre dossier contiendra les dossiers/fichiers suivants :
\begin{code}
    \begin{minted}[breaklines,breaksymbol=,linenos,frame=single,stepnumber=1,tabsize=2]{yaml}
- Roles
    - "Nom du role"
        - Defaults : les variables par défaut qui seront à disposition du rôle.
            - main.yml
        - Vars : Variables à disposition du rôle cependant elles ont vocation à être modifiées par l'utilisateur et elles prennent le dessus sur celle du dossier « defaults » si elles sont renseignées.
            - main.yml
        - Tasks : ici on renseigne nos taches (comme dans un playbooks normal).
            - main.yml
        - Meta : Sert à renseigner les dépendances liées à nos rôles (ssl, et etc).
            - main.yml
        - README : renseigne sur comment utilisé les rôles, variables à définir et etc.
    \end{minted}
    \caption{Gestion des rôles}
    \label{gestion_roles}
\end{code}
\bigbreak
On appelle un rôle de la façon suivante:
\begin{code}
    \begin{minted}[breaklines,breaksymbol=,linenos,frame=single,stepnumber=1,tabsize=2]{yaml}
---
  tasks:
- include_role: role.yml
    \end{minted}
    \caption{Appel d'un rôle}
    \label{appel_role}
\end{code}
\bigbreak

\subsection{Bibliographie}
\begin{itemize}
    \item \url{https://github.com/ansible/ansible/issues/19584}
    \item \url{https://docs.ansible.com/ansible/latest/yum_module.html}
    \item \url{https://docs.ansible.com/ansible/latest/service_module.html}
    \item \url{https://docs.ansible.com/ansible/latest/apt_module.html}
    \item \url{https://docs.ansible.com/ansible/latest/apt_repository_module.html}
    \item \url{https://serversforhackers.com/c/an-ansible-tutorial}
    \item \url{https://buzut.fr/tirer-toute-puissance-dansible-roles/}
\end{itemize}

\end{document}
